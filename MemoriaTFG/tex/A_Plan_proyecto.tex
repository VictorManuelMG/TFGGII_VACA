\apendice{Plan de Proyecto Software}

\section{Introducción}

Este plan detalla los aspectos clave en la planificación, gestión y viabilidad del desarrollo del software \textbf{Voice-Assisted Computer Accessibility (VACA)}.

El proyecto es una solución accesible y funcional que facilita el uso de sistemas informáticos a personas con movilidad reducida, introduciendo tecnologias de voz, visión artifical e inteligencia artifical.

Este documento incluye una planificacion temporal asi como un analisis de la viabilidad economica y legal.

\section{Planificación temporal}

La planificación del proyecto se realizó siguiendo una metodología ágil, dividiendo el trabajo en sprints que cubren diferentes fases. El proyecto abarca desde febrero hasta junio, organizándose en las siguientes etapas:

\begin{enumerate}
    \item \textbf{Fase de investigación (febrero)}: Revisión del estado del arte de CUAs, estudio de modelos STT/TTS, aprendizaje sobre frameworks para el uso de LLMs, herramientas para visión artifical.
    
    \item \textbf{Desarrollo del back-end (marzo-abril)}:
    Integración de LLMs, modelos de ASR, modelos de visión, y arquitectura entre todas las partes a un agente principal.
    
    \item \textbf{Desarrollo del front-end (abril-mayo)}: 
    Busqueda de una herramienta OpenSource para el desarrollo de la interfaz gráfica.

    Uso y aprendizaje de Tkinter y posterior paso a CTkinter integrando back-end y front-end.
    
    \item \textbf{Pruebas y validación (mayo)}: Evaluación funcional de las herramientas, pruebas de prompts de voz, tiempos de respuesta de los modelos de inferencia, pruebas del software en local/VM.
    
    \item \textbf{Documentación (mayo-junio)}: elaboración de una memoria del proyecto en conjunto con los anexos del mismo.
\end{enumerate}

\section{Estudio de viabilidad}

\subsection{Viabilidad económica}

El proyecto se ha desarrollado utilizando principalmente herramientas y recursos gratuitos o de código abierto, lo cual ha permitido minimizar los costes asociados. A continuación, se detallan los principales aspectos económicos:

\begin{itemize}
    \item \textbf{Herramientas utilizadas:} Python, CTkinter, OpenCV, PyAutoGUI, LangChain, Whisper, COQUI-TTS, Docker y FastAPI. Todas ellas son gratuitas y de código abierto.
    
    \item \textbf{Recursos computacionales:} Durante el desarrollo se utilizó hardware local, y en fases avanzadas, modelos pesados como YOLOv8 y FlorenceV2 fueron desplegados en un servidor del ITCL equipado con una \textbf{GPU NVIDIA A30 de 24GB}, optimizando así los tiempos de inferencia.

    \item \textbf{Modelos LLM:} Se emplearon tanto modelos de código abierto (FlorenceV2, YOLOv8, Whisper, COQUI) como APIs comerciales (Claude y GPT-4) que implicaron costes variables.
\end{itemize}

\vspace{1em}
A continuación, se desglosan los costes estimados si el sistema se llevase a producción real, considerando tanto infraestructura como licencias, consumo energético y otros factores de operación continua.

\begin{table}[H]
	\centering
	\renewcommand{\arraystretch}{1.5}
	\rowcolors{2}{gray!20}{white}
	\resizebox{\textwidth}{!}{
		\begin{tabular}{m{8cm} >{\centering\arraybackslash}m{3cm} >{\centering\arraybackslash}m{3cm}}
			\toprule
			\textbf{Elemento} & \textbf{Coste estimado} & \textbf{Frecuencia} \\
			\midrule
			Servidor con GPU NVIDIA A30 (24GB VRAM) & 6.000 € & Único \\
			Consumo eléctrico servidor (8h/día) & 14,40 € & Mensual \\
			Conexión a internet (fibra 1 Gbps) & 40 € & Mensual \\
			Mantenimiento de infraestructura (ITCL) & 600 € & Anual \\
			\bottomrule
		\end{tabular}
	}
	\caption{Coste estimado de infraestructura y recursos computacionales}
	\label{tab:coste_infraestructura}
\end{table}

\begin{table}[H]
	\centering
	\renewcommand{\arraystretch}{1.5}
	\rowcolors{2}{gray!20}{white}
	\resizebox{\textwidth}{!}{
		\begin{tabular}{m{8cm} >{\centering\arraybackslash}m{3cm} >{\centering\arraybackslash}m{3cm}}
			\toprule
			\textbf{Servicio/API} & \textbf{Coste estimado} & \textbf{Frecuencia} \\
			\midrule
			OpenAI GPT-4 (100.000 tokens/mes aprox.) & 20 € & Mensual \\
			Anthropic Claude (acceso de desarrollador) & 15 € & Mensual \\
			\bottomrule
		\end{tabular}
	}
	\caption{Costes asociados al uso de APIs comerciales de modelos LLM}
	\label{tab:coste_llm}
\end{table}


\begin{table}[H]
	\centering
	\renewcommand{\arraystretch}{1.5}
	\rowcolors{2}{gray!20}{white}
	\resizebox{\textwidth}{!}{
		\begin{tabular}{m{8cm} >{\centering\arraybackslash}m{3cm} >{\centering\arraybackslash}m{3cm}}
			\toprule
			\textbf{Elemento} & \textbf{Coste estimado} & \textbf{Frecuencia} \\
			\midrule
			Salario programador (5 meses) & 7.500 € & Proyecto \\
			Hardware usuario final (PC + mic + auriculares) & 500 € & Único \\
			\bottomrule
		\end{tabular}
	}
	\caption{Otros costes estimados para desarrollo y operación}
	\label{tab:coste_otros}
\end{table}

\textbf{Alternativa económica basada en ejecución local:}

Como alternativa al mantenimiento continuo de un servidor con GPU dedicada, se considera viable la ejecución del sistema en local utilizando un ordenador personal de gama media-alta con una GPU dedicada. Esto permitiría realizar inferencias de modelos como YOLOv8 o Florence sin necesidad de recurrir a infraestructura externa, reduciendo así los costes mensuales de operación.

A continuación, se muestra una configuración de hardware recomendada para lograr tiempos de inferencia aceptables ejecutando el sistema en local:


\begin{table}[H]
	\centering
	\renewcommand{\arraystretch}{1.5}
	\rowcolors{2}{gray!20}{white}
	\resizebox{\textwidth}{!}{
		\begin{tabular}{m{8cm} >{\centering\arraybackslash}m{3cm} >{\centering\arraybackslash}m{3cm}}
			\toprule
			\textbf{Elemento} & \textbf{Coste estimado} \\
			\midrule
			PC Case & 80 €  \\
			Placa Base (MSI B550M PRO-VDH WIFI & 111 € \\
            RAM DDR4 3200Mhz 32GB & 60 € \\
            Procesador AM4 (Ryzen 7 5800X) & 180 € \\
            Refrigeración & 80 € \\
            PSU 750W & 100€ \\
            GPU 6GB VRAM (RTX 3050) & 210€ \\
            Disco Duro & 100 €\\
			\bottomrule
		\end{tabular}
	}
	\caption{Coste de un ordenador que pueda usar el modelo en local sin servidores}
	\label{tab:coste_otros}
\end{table}

\textbf{Resumen económico aproximado del proyecto en fase de producción:}


\begin{itemize}
    \item \textbf{Coste inicial de la infraestructura:} Incluye la adquisición de un servidor con GPU (NVIDIA A30), asi como el hardware minimo necesario para un usuario final, el coste estimado asciende a un valor de entre \textbf{6000 - 8000 €}.

    \item \textbf{Coste operativo mensual:} Este se comprende del consumo electrico del servidor, conexión a internet, uso de APIs comerciales de OpenAI y Anthropic. Esto tendria un coste aproximado de \textbf{80-100€ mensuales}

    \item \textbf{Coste de desarrollo humano:} Estimado de un único programador trabajando durante 5 meses, con una media salarial de 1500€ mensuales. \textbf{Total: 7500€}
\end{itemize}


Considerando el uso intensivo de herramientas open-source, la reutilización de infraestructura preexistente en el ITCL y la posibilidad de sustituir APIs de pago por modelos open-source (cambiando APIs de Claude y OpenAI), el proyecto presenta una \textbf{viabilidad económica alta}. Puede ser escalado o adaptado a distintos entornos sin incurrir en costes prohibitivos, lo que lo convierte en una solución sostenible tanto a corto como a largo plazo.


\subsection{Viabilidad legal}

El proyecto cumple con las normativas legales vigentes en los siguientes aspectos:

\begin{itemize}
    \item \textbf{Licencias de software:} Todas las herramientas utilizadas (Python, bibliotecas de IA, modelos open-source) se encuentran bajo licencias compatibles con su uso, modificación y distribución (MIT, Apache 2.0, GNU).
    
    \item \textbf{Protección de datos personales:} El sistema no almacena información personal ni sensible. Las grabaciones de voz son procesadas de forma local o en servidores seguros del ITCL, sin ser enviadas a terceros, ademas de que los contenedores de los modelos son destruidos tras un tiempo de inactividad sin dejar ningun tipo de dato en el servidor.
    
    \item \textbf{Accesibilidad:} El software está diseñado específicamente para cumplir con principios de accesibilidad digital, lo que alinea con las directrices de la \textit{Ley General de Discapacidad}.
    
    \item \textbf{Código abierto:} El código desarrollado puede ser compartido, auditado o ampliado por terceros, fomentando la transparencia, colaboración y reutilización del software en contextos sociales o institucionales.

\end{itemize}

\begin{table}[H]
	\centering
	\renewcommand{\arraystretch}{1.5}
	\rowcolors{2}{gray!20}{white}
	\resizebox{\textwidth}{!}{
		\begin{tabular}{m{7cm} >{\centering\arraybackslash}m{4cm} >{\centering\arraybackslash}m{4cm}}
			\toprule
			\textbf{Requisito Legal} & \textbf{Aplicación en VACA} & \textbf{Cumplimiento} \\
			\midrule
			Accesibilidad universal & Control por voz sin necesidad de dispositivos físicos & Sí \\
			Diseño para todos (diseño universal) & Accesible sin adaptaciones adicionales & Sí \\
			Acceso a las TICs & Compatible con lectores de pantalla y salida TTS & Sí \\
			Solución no discriminatoria & Uso gratuito, sin hardware costoso & Sí \\
			Accesibilidad desde el diseño & Interfaz visual adaptada desde fase inicial & Sí \\
			\bottomrule
		\end{tabular}
	}
	\caption{Cumplimiento del proyecto VACA con la Ley General de Discapacidad (España)}
	\label{tab:ley_discapacidad}
\end{table}


\subsection{Amortización del proyecto}

Teniendo en cuenta los costes descritos previamente, se estima que el coste total de desarrollo y puesta en marcha del sistema \textbf{VACA} ronda los \textbf{8.000 a 10.000 €}, incluyendo infraestructura, desarrollo, y coste de APIs comerciales en el primer año.

Es importante destacar que parte del hardware utilizado, especialmente el servidor con GPU proporcionado por el ITCL, no está dedicado exclusivamente a este proyecto. Dicho servidor alberga modelos que también se utilizan en otras investigaciones y aplicaciones, lo que permite distribuir parte del coste de infraestructura entre varios desarrollos. Esta reutilización de recursos mejora significativamente la rentabilidad global del sistema.

Aunque el código fuente de VACA se encuentra alojado públicamente en GitHub como \textbf{software libre}, se podria contemplar su distribución mediante versiones comerciales mantenidas y actualizadas, orientadas a usuarios con movilidad reducida. Estas versiones podrían incluir soporte técnico y actualizaciones periódicas.


\subsubsection{Resumen estimado de amortización:}

\begin{itemize}
    \item \textbf{Coste inicial total:} 10.000 € aprox.\ (con parte de infraestructura compartida)
    \item \textbf{Coste de mantenimiento durante 5 años:} 600 € × 5 = 3.000 €
    \item \textbf{Coste total acumulado estimado para VACA:} \textbf{13.000 €}, aunque parte del hardware puede considerarse amortizado parcialmente por su uso compartido.
\end{itemize}

Si el sistema fuese licenciado a \textbf{50 usuarios} durante esos cinco años con una tarifa de \textbf{5 € mensuales} por usuario (modelo de suscripción), se obtendría un ingreso de:

\[
50 \text{ usuarios} \times 5 \text{ €/mes} \times 12 \text{ meses} \times 5 \text{ años} = \textbf{15.000 €}
\]

Este escenario permitiría no solo amortizar completamente la inversión realizada, sino también generar un pequeño margen económico. En caso de alcanzar una mayor base de usuarios o integrar el sistema en contextos institucionales o sanitarios, el retorno económico sería aún más favorable.

Adicionalmente, el sistema sustituye soluciones comerciales mucho más costosas, y reduce significativamente la necesidad de hardware especializado, haciendo su adopción más accesible y sostenible a nivel económico y social.

\subsubsection{Conclusión} 

La amortización del proyecto es viable incluso en contextos de adopción moderada. El uso compartido de infraestructura, el enfoque open-source y la orientación social del sistema refuerzan su sostenibilidad técnica y económica en el medio y largo plazo.

