\apendice{Anexo de sostenibilización curricular}

\section{Introducción}

Este anexo tiene como objetivo hacer una reflexión personal sobre cómo se han abordado temas de sostenibilidad a lo largo del desarrollo de este Trabajo de Fin de Grado (TFG). A través del proyecto \textbf{VACA} (Voice Assisted Computer Accessibility), he podido trabajar no solo con herramientas técnicas, sino también con ideas relacionadas con la accesibilidad, la inclusión, el uso responsable de la tecnología y el impacto que puede tener un software en el entorno y en las personas.

\section{Contribución del TFG a la inclusión social}

Desde el principio, uno de los objetivos principales del proyecto fue crear una herramienta que pudiera ayudar a personas con movilidad reducida a interactuar con su ordenador de forma más sencilla. \textbf{VACA} permite controlar el equipo mediante comandos de voz, algo que puede marcar la diferencia para muchos usuarios que no pueden utilizar el teclado o el ratón con normalidad.

Esto conecta directamente con la idea de reducir desigualdades, tal como plantea el Objetivo de Desarrollo Sostenible (ODS) número 10 de la ONU. Me parece importante que desde la informática se puedan proponer soluciones que tengan un impacto positivo en la vida de la gente, especialmente en colectivos que suelen quedar fuera del foco tecnológico.

\section{Sostenibilidad tecnológica y uso de software libre}

Durante todo el desarrollo he apostado por el uso de herramientas y modelos de código abierto como Whisper, Coqui o Florence. Además de ser gratuitos, estos recursos están disponibles para toda la comunidad, lo que permite que otras personas puedan aprovecharlos, mejorarlos o adaptarlos a sus propios proyectos.

Trabajar con software libre me ha hecho más consciente del valor de compartir el conocimiento, y también de cómo esto ayuda a reducir barreras económicas. No depender de soluciones cerradas o de pago también puede ser una forma de hacer la tecnología más accesible para más gente, y eso encaja perfectamente con la idea de sostenibilidad social y económica.

\section{Impacto ambiental y decisiones técnicas}

Aunque muchas veces no lo pensemos, el software también tiene un impacto ambiental. Elegir si un modelo se ejecuta en local o en la nube puede influir en el consumo energético. En mi caso, he intentado que la mayoría de procesos puedan hacerse de forma local, sin necesidad de servidores externos. Esto no solo hace que el programa sea más rápido y autónomo, sino que también evita depender de infraestructuras que consumen muchos recursos.

También he intentado que el sistema no esté constantemente ejecutando procesos pesados si no son necesarios, lo que ayuda a optimizar el rendimiento y reducir el gasto de energía.

\section{Competencias de sostenibilidad adquiridas}

Durante el proyecto he aprendido muchas cosas, no solo técnicas, sino también relacionadas con la sostenibilidad. Por ejemplo, ahora tengo más claro que pensar en la accesibilidad no es solo un añadido, sino algo que debería formar parte del diseño desde el principio.

También he desarrollado una mirada más crítica a la hora de tomar decisiones técnicas. Ya que no se trata solo de incluir modelos sin gestión propia, sino también de que implicaría usar dicho modelo, si es abierto, si deja huella en internet, etc.

Por último, he podido ver cómo la informática puede ser una herramienta útil para cambiar cosas en el mundo real. A veces damos por hecho que los proyectos técnicos solo sirven para resolver problemas funcionales, pero también pueden tener un impacto social muy potente.

\section{Conclusión}

Este TFG me ha ayudado a tomar conciencia de cómo el desarrollo de software puede (y debería) hacerse teniendo en cuenta aspectos éticos y sostenibles. A lo largo del trabajo he intentado aplicar este enfoque en lo que estaba a mi alcance, y me ha motivado a seguir haciéndolo en el futuro.

Mi intención es seguir trabajando con esta mentalidad en los proyectos que vengan, combinando lo técnico con lo humano. Creo que la informática tiene un gran potencial para mejorar la vida de las personas, y que es responsabilidad nuestra como desarrolladores usar ese potencial de forma responsable y consciente.

En definitiva, este proyecto ha sido mucho más que una práctica técnica: me ha permitido abrir los ojos a un enfoque más completo y comprometido con la sociedad.