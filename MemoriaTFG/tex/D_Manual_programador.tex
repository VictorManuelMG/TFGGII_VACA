\apendice{Documentación técnica de programación}

\section{Introducción}

\section{Datos tratados}


\section{Estructura de directorios}

\begin{itemize}
  \item \textbf{Raíz del proyecto (VACA/)}
  \begin{itemize}
    \item \texttt{main\_loop.py} -- Bucle principal del agente
    \item \texttt{gui.py} -- Interfaz gráfica en CTkinter
    \item \texttt{initializer.py} -- Inicialización del entorno
    \item \texttt{.env}, \texttt{pyproject.toml}, \texttt{README.md}
  \end{itemize}
  
  \item \textbf{CUA/} -- Módulo central del agente
  \begin{itemize}
    \item \texttt{cfg.py}, \texttt{\_\_init\_\_.py}
    \item \textbf{tools/}
    \begin{itemize}
      \item \texttt{class\_whisper.py}, \texttt{class\_browser\_use.py}, \texttt{computer.py}
      \item \texttt{endpoint\_yolo\_florence.py} -- Objetos de los modelos para instanciación y tools varias.
      \item \texttt{audio\_TTS.wav}, \texttt{output.wav} -- Generacion de TTS y STT
      \item \textbf{tmpcrops/} -- Capturas temporales para detección de elementos visuales
    \end{itemize}
    \item \textbf{util/} -- Funciones de soporte (configuración, rutas, logs)
  \end{itemize}

  \item \textbf{assets/} -- Material gráfico del proyecto
  \begin{itemize}
    \item Diagramas: \texttt{CUA FLOW.png}, \texttt{GUI\_inicial.png}, \texttt{resultado\_final.png}
    \item Pruebas visuales: \texttt{ejemplo\_original.jpeg}, \texttt{ejemplo\_yoloed.jpeg}
    \item Videos demostrativos: \texttt{cua\_example1.mp4}, \texttt{cua\_example2.mp4}
  \end{itemize}

  \item \textbf{tests/} -- Suite de pruebas del sistema
  \begin{itemize}
    \item \texttt{test\_001\_ScreenAssistant.py}, \texttt{test\_002\_Florence.py}, etc.
    \item \textbf{resources/}
    \begin{itemize}
      \item \textbf{generatedAudio/} -- Audios generados en pruebas
      \item \textbf{generatedimage/} -- Imágenes procesadas por los modelos
    \end{itemize}
  \end{itemize}

  \item \textbf{logs/} -- Registro de eventos y configuración
  \begin{itemize}
    \item \texttt{CUA.util.cfg\_base.log}
  \end{itemize}
\end{itemize}


\section{Manual del programador}

En esta sección elaborare una "visión" técnica para que cualquier otra persona que desee comprender, mantener o mejorar la aplicación VACA.

A continuación detallare los módulos clave y las dependendias principales del proyecto:

\begin{itemize}
    \item \textbf{Lenguaje de programación:} Python 3.12
    \item \textbf{Gestor de dependencias:} Poetry
    \item \textbf{Frameworks y librerías clave:}
    \begin{itemize}
        \item \texttt{openai, langchain-openai, langchain-anthropic,python-dotenv}
        \item \texttt{pyautogui, pillow,anthropic,langchain}
        \item \texttt{opencv-python, screeninfo, ultralytics,safetensors}
        \item \texttt{transformers,langchain-community,keyboard}
        \item \texttt{timm,einops,langgraph,logging}
        \item \texttt{pyaudio,wave,browser-use}
        \item \texttt{threaded}
    \end{itemize}
    \item \textbf{Estructura modular:}
    \begin{itemize}
        \item Módulo \texttt{CUA/}: lógica del agente, herramientas de audio y visión.
        \item \texttt{main\_loop.py}: orquestador del flujo principal.
        \item \texttt{gui.py}: interfaz visual del sistema.
        \item \texttt{initializer.py}: carga inicial para levantar modelos del servidor del ITCL.
    \end{itemize}
\end{itemize}

\subsection{Instrucciones para ejecución sin modelos dockerizados}

En caso de querer ejecutar la aplicación sin utilizar los modelos dockerizados para \texttt{Whisper}, \texttt{Coqui}, \texttt{Florence} o \texttt{YOLO}, se proporciona en el repositorio de GitHub el contenedor Docker correspondiente al modelo \texttt{Florence-YOLO}, facilitando así su despliegue local. Para ello, únicamente sería necesario adaptar los endpoints a la configuración local deseada.

Respecto a los modelos \texttt{Whisper} y \texttt{Coqui}, no se incluyen directamente en el proyecto debido a motivos dep privacidad. Sin embargo, ambos modelos son de código abierto y están disponibles en los siguientes enlaces:

\begin{itemize}

\item \textbf{Coqui XTTS-v2}\cite{coquiXTTSv2}
\item \textbf{OpenAI Whisper Large v3}\cite{openaiWhisperV3}
\end{itemize}

Para ejecutar el sistema sin estos modelos, se recomienda desactivar el módulo de TTS (activado por defecto) y también el hilo de transcripción continua (STT) que se lanza desde la clase \texttt{GUI}. Esta configuración permitirá el uso parcial del agente sin necesidad de contar con los modelos de voz en funcionamiento, manteniendo otras funcionalidades operativas.





\section{Compilación, instalación y ejecución del proyecto}

En esta sección se describen los pasos necesarios para compilar, instalar y ejecutar correctamente la aplicación \textbf{VACA}.

\subsection{Descarga del programa desde GitHub}

El código del proyecto se encuentra disponible en el repositorio personal del alumno. Se recomienda descargar la última versión publicada en la sección de releases del repositorio.\cite{VACARepo}

\subsection{Instalación de los programas necesarios e inicialización}

Para la correcta ejecución del proyecto, se deben seguir los siguientes pasos:

\begin{enumerate}
    \item Instalar un entorno de desarrollo para Python. En este caso se ha utilizado Visual Studio Code, aunque puede utilizarse cualquier otro compatible.
    \item Instalar Python en su versión 3.12. \cite{python312}
    
    \item Instalar un gestor de entornos virtuales compatible con archivos \texttt{pyproject.toml}. En este proyecto se ha utilizado \textbf{Poetry}\cite{poetryDocs}.
    
    \item Crear un nuevo entorno virtual y utilizar el archivo \texttt{pyproject.toml} del repositorio para importar automáticamente todas las librerías necesarias.
    
    \item Una vez finalizada la instalación de dependencias, se puede ejecutar el archivo \texttt{gui.py}, el cual lanza la interfaz principal del programa.
\end{enumerate}

\subsection{Recomendaciones de uso}

Se recomienda encarecidamente utilizar un sistema con doble monitor, ya que la aplicación trabaja principalmente sobre el monitor configurado como principal, y requiere visibilidad total del entorno gráfico para su funcionamiento correcto.

\section{Pruebas del sistema}

El sistema fue sometido a diversas pruebas con el objetivo de garantizar su correcto funcionamiento en distintos contextos de uso.

Dado que \textbf{VACA} se basa en el uso de modelos de lenguaje de gran escala (LLM) y controla el entorno operativo del usuario, muchas de las pruebas fueron realizadas de forma manual. Esto se debe a que, por la naturaleza del sistema, cualquier ejecución fuera de control podría requerir intervención humana para ser detenida o corregida.

\subsection{Pruebas automatizadas}

Se implementó una batería de pruebas unitarias y funcionales que puede encontrarse en la carpeta \texttt{test} del repositorio del proyecto. Estas pruebas tienen como objetivo verificar el correcto funcionamiento individual de los distintos módulos que componen el sistema.

Para ejecutar todas las pruebas, basta con utilizar el siguiente comando en el entorno del proyecto:

\texttt{pytests \@master\_test.txt}

Este comando lanza la ejecución de todos los casos definidos de manera estructurada.

Adicionalmente, se valoró la posibilidad de integrar la aplicación en un entorno de evaluación tipo \textit{benchmark} con el fin de analizar su rendimiento de forma más rigurosa. Sin embargo, debido a que se trata de un agente de "reciente" y que funciona exclusivamente sobre el sistema operativo Windows, no existen muchas opciones disponibles.

La única alternativa identificada fue el entorno de evaluación \textbf{Windows Agent Arena} \cite{Bonatti2024WindowsAgentArena}. No obstante, su integración se descartó en esta fase del proyecto debido a la falta de tiempo y la complejidad técnica para su puesta en marcha, quedando como posible mejora a implementar en el futuro.
