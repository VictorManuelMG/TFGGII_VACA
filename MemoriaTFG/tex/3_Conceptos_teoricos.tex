\capitulo{3}{Conceptos teóricos}

En este capítulo se presentan los conceptos teóricos clave que sustentan el desarrollo del proyecto \textbf{Voice-Assisted Computer Accessibility (VACA)}. Estos conceptos son fundamentales para entender el funcionamiento del sistema, las tecnologías empleadas y la problemática a la que se enfrenta. A continuación, se explican los principales términos relacionados con el \textbf{Computer Use Agents (CUA)}, \textbf{speech-to-text (SST)}, \textbf{text-to-speech (TTS)} y otros métodos empleados en la creación del software.

Estas tecnologías, aun que son diversas en naturaleza entre sí, deben integrarse sinergicamente entre ellas para proporcionar una experiencia fluida. Comprender el funcionamiento individual de estas permitira entender los retos a los que me enfrentare a la hora de la realización del proyecto.

\section{Agentes de uso de computadora (CUA)}

Los \textbf{Computer Use Agents (CUA)} son sistemas inteligentes diseñados para asistir a los usuarios en tareas específicas a través de interfaces graficos, usando los recursos disponibles del sistema, automatizando asi ciertas tareas que realizaria un usuario.

En el contexto de este proyecto, los CUA se utilizan para controlar el ordenador mediante prompts de voz haciendo asi el proceso de controlar un sistema mas accesible y eficiente para personas con movilidad reducida.

\section{Speech-to-Text (STT)}

El \textbf{Speech-to-Text} (STT) es una tecnología que convierte las palabras habladas en texto escrito. Este proceso se logra mediante el uso de algoritmos de reconocimiento de voz, que permiten a las máquinas comprender el lenguaje hablado. SST es una herramienta clave en este proyecto, ya que permite que los usuarios controlen el sistema informático sin necesidad de usar el teclado o el ratón, lo que facilita la accesibilidad para personas con movilidad reducida.

\section{Text-to-Speech (TTS)}

El \textbf{Text-to-Speech} (TTS) es una tecnología que convierte texto escrito en voz. Esta tecnología es crucial para proporcionar retroalimentación auditiva a los usuarios, permitiéndoles interactuar con el sistema de manera más eficiente. En este proyecto, el TTS se usa para leer instrucciones, mensajes de error o cualquier otra información relevante para el usuario.


\section{Inteligencia Artificial (IA) y Agentes Inteligentes}

La \textbf{Inteligencia Artificial} (IA) se refiere a la simulación de procesos de inteligencia humana mediante algoritmos y sistemas computacionales. En el contexto de este proyecto, la IA se utiliza para crear \textbf{agentes inteligentes} capaces de aprender y adaptarse a las necesidades específicas de los usuarios. Estos agentes son esenciales para ofrecer una experiencia personalizada y eficiente, ajustando el comportamiento del software a las características individuales de cada usuario.

\section{Modelos de Lenguaje Grande (LLM)}

Los \textbf{Modelos de Lenguaje Grande} son sistemas basados en inteligencia artificial que permiten a las máquinas comprender y generar texto de manera coherente. En este proyecto, los modelos de lenguaje son fundamentales para que los \textbf{agentes inteligentes} comprendan y respondan a los prompts de voz de los usuarios. Estos modelos mejoran la precisión y adaptabilidad del sistema, permitiendo una interacción más fluida y personalizada, especialmente para personas con movilidad reducida.

\section{Modelos multimodales}

Los \textbf{modelos multimodales} son sistemas de inteligencia artificial capaces de procesar y relacionar información proveniente de múltiples modalidades, como texto, voz, imágenes y video. Estos modelos integran distintos tipos de datos para comprender contextos complejos y ofrecer respuestas más precisas y contextualizadas. En el proyecto VACA, los modelos multimodales permiten combinar prompts de voz con información visual en pantalla, mejorando así la interacción entre el usuario y el sistema. Esta capacidad es totalmente necesaria para que el programa pueda ser consciente de lo que quiere transmitir el usuario y entender el entorno gráfico del usuario.

\section{Modelos de visión}

Los \textbf{modelos de visión por computadora} son algoritmos diseñados para interpretar y analizar imágenes o secuencias visuales, permitiendo que los sistemas comprendan elementos visuales como ventanas, botones, íconos o texto en pantalla. En el proyecto VACA, estos modelos seran utilizados para identificar componentes gráficos del sistema operativo y permitir al \textbf{CUA} ser consciente de lo que le rodea en el entorno gráfico.


 