\capitulo{2}{Objetivos del proyecto}

El proyecto \textbf{Voice-Assisted Computer Accessibility (VACA)} tiene como propósito principal mejorar la accesibilidad de las personas con movilidad reducida en el uso de sistemas informáticos. A través de la integración de tecnologías de \textbf{speech-to-text (SST)}, \textbf{text-to-speech (TTS)}, y \textbf{agentes inteligentes}, se busca desarrollar una solución accesible y eficiente que permita a los usuarios controlar el ordenador mediante prompts de voz y asistencia basada en inteligencia artificial.

\section{Objetivos específicos}

\begin{itemize}
    \item \textbf{Desarrollar un software accesible} que permita a personas con movilidad reducida utilizar un ordenador sin necesidad de dispositivos físicos adicionales, utilizando solo prompts de voz y agentes inteligentes.
    
    \item \textbf{Integrar tecnologías de Speech-to-Text (SST)} para convertir el habla en texto y permitir que los usuarios controlen el sistema mediante prompts de voz.
    
    \item \textbf{Implementar la tecnología Text-to-Speech (TTS)} para proporcionar retroalimentación auditiva, mejorando la interacción del usuario con el sistema.
    
    \item \textbf{Incorporar agentes inteligentes basados en Inteligencia Artificial} \textbf{(IA)} que proporcionen asistencia personalizada y adaptativa a las necesidades individuales de los usuarios, permitiendo una mayor eficiencia en el control del ordenador.
    
    \item \textbf{Reducir la dependencia de hardware especializado} mediante el uso de soluciones basadas en software, lo que facilita el acceso a personas con movilidad reducida a un costo más bajo.
    
    \item \textbf{Evaluar la precisión del reconocimiento de voz } mediante pruebas de usabilidad, a fin de garantizar que el sistema cumpla con los requisitos de accesibilidad y facilidad de uso.

    \item \textbf{Evaluar la precisión del CUA} mediante pruebas de usabilidad, afinar el reconocimiento grafico del LLM con el entorno del sistema.
    
    \item \textbf{Realizar un despliegue efectivo del software,} permitiendo que los usuarios finales puedan instalar y utilizar el sistema en entornos de Windows.
    
\end{itemize}


\section{Objetivos adicionales a futuro.}

\begin{itemize}
    \item \textbf{Soporte multilingüe:} Incorporar soporte para varios idiomas en el software, permitiendo que usuarios de diferentes partes del mundo puedan beneficiarse del sistema, adaptando el sistema de reconocimiento de voz y las respuestas TTS a distintos idiomas.
    
    \item \textbf{Interfaz de usuario adaptativa:} Desarrollar una interfaz gráfica de usuario (GUI) que se adapte a las necesidades y preferencias de cada usuario.
    
    \item \textbf{Compatibilidad con sistemas operativos adicionales:} Ampliar la compatibilidad del software para que sea accesible en otros sistemas operativos además de Windows, como macOS o Linux.
    
    \item \textbf{Funcionalidad de aprendizaje automático para mejorar la precisión del reconocimiento de voz:} Implementar un sistema de retroalimentación en tiempo real que permita al software aprender y adaptarse a las peculiaridades del habla del usuario, mejorando continuamente la precisión del reconocimiento de voz mediante el uso de algoritmos de aprendizaje automático.

    \item \textbf{Reduccion de costo de los modelos de lenguaje} utilizando prompt engineering para minimizar el uso de tokens, así mismo el costo del modelo, o integrando modelos de lenguaje de código abierto.

    \item \textbf{Implementación de una solución compacta y portatil} basandose en Jetson Orin, para proteger la privacidad y autonomía del usuario, permitiendo una gestión completamente local y segura de su información.
\end{itemize}


\section{Objetivos personales}

\begin{itemize}

    \item\textbf{Formacion en LLM:} Como a futuro me gustaría llevar mi carrera profesional en la dirección de la rama de inteligencia artifical esto podría ser un buen comienzo y una buena toma de contacto con estas tecnologías.
    
    \item\textbf{Desarrollo de una aplicación con futuro de producción:} Poder crear una aplicación para futuro uso del ITCL me dará la oportunidad de enfocarme en detalles como la escalabilidad y modularidad del proyecto.
    
    \item\textbf{Dejar una huella en la empresa ITCL:} Dejar una marca por mi paso por el ITCL me haría ilusión y dejandoles este proyecto y el conocimiento de estas herramientas en sus manos, sería un gran logro para esto.
    
\end{itemize}